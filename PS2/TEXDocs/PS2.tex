\documentclass{article}
\addtolength{\oddsidemargin}{-.875in}
\addtolength{\evensidemargin}{-.875in}
\addtolength{\textwidth}{1.75in}
\addtolength{\topmargin}{-.875in}
\addtolength{\textheight}{1.75in}

\usepackage{amsmath}
\usepackage{rotating}
\usepackage{booktabs}
\usepackage{dcolumn}

\begin{document}

\begin{center}
\textbf{ECON 2320 -- Economics of Labor and Population} \\
\textbf{Problem Set 2} \\
Samuel Brown \\
Fall 2011 \\
Due: October 26

\end{center}
\bigskip

\arraycolsep 1pt

\begin{enumerate}
\item \begin{enumerate}

\item To identify the average treatment effect of maternal smoking by comparing unadjusted group means, smoking must be randomly assigned to mothers. An estimate of the effect of smoking on birthweight using this method can be seen in the first column of Table~\ref{reg:ols}. The mean characteristics of non-smokers and smokers can be seen in Table~\ref{tab:groupm}. As is evident from these tables, non-smokers and smokers exhibit significantly different observable characteristics. This is evidence against random assignment of smoking.

\begin{table}[htbp]\centering
\def\sym#1{\ifmmode^{#1}\else\(^{#1}\)\fi}
\caption{Differences in Sample Means Between Groups of Counties, Defined by TSPs Levels or Nonattainment Status\label{tab:groupm}}
\begin{tabular}{l*{2}{D{.}{.}{-1}}}
\toprule
                    &\multicolumn{1}{c}{(1)}&\multicolumn{1}{c}{(2)}\\
                    &\multicolumn{1}{c}{\parbox{1.5in}{\centering First Difference 1980--1970}}&\multicolumn{1}{c}{\parbox{1.5in}{\centering TSPs Nonattainment in 1975 or 1976}}\\
\midrule
Housing value       &   -3166.017\sym{***}&    2621.042\sym{***}\\
                    &   (705.745)         &   (789.447)         \\
\addlinespace
Mean TSPs           &     -30.956\sym{***}&      -9.854\sym{***}\\
                    &     (1.020)         &     (1.544)         \\
\addlinespace
Income per capita (1982--84 dollars)&    -158.767\sym{***}&      48.012         \\
                    &    (40.525)         &    (45.448)         \\
\addlinespace
Unemployment rate ($\times 100$)&       0.522\sym{***}&       0.031         \\
                    &     (0.129)         &     (0.144)         \\
\addlinespace
\% employment in manufacturing ($\times 10$)&      -0.112\sym{***}&      -0.005         \\
                    &     (0.026)         &     (0.029)         \\
\addlinespace
Population density  &     -66.859\sym{***}&     -19.050         \\
                    &    (24.562)         &    (27.446)         \\
\addlinespace
\% urban ($\times 10$)&      -0.007         &      -0.001         \\
                    &     (0.005)         &     (0.006)         \\
\addlinespace
\% houses build in last 10 years&      -0.034\sym{***}&      -0.007         \\
                    &     (0.007)         &     (0.008)         \\
\midrule
\ensuremath{N}      &\multicolumn{1}{c}{1000}         &\multicolumn{1}{c}{1000}         \\
\bottomrule
\multicolumn{3}{l}{\footnotesize Mean values; standard errors in parentheses.}\\
\multicolumn{3}{l}{\footnotesize \sym{*} \(p<0.10\), \sym{**} \(p<0.05\), \sym{***} \(p<0.01\)}\\
\end{tabular}
\end{table}


\item An adjusted estimate of the average treatment effect of smoking on birthweight can be seen in the second column of Table~\ref{reg:ols}. This effect is identified if maternal smoking is randomly assigned conditional on the observables, viz., all unobservable determinants of birthweight are {\em independent} of smoking assignment, conditional on the observables.

\item The restrictive linear structure of (b) is likely a source of specification bias. An estimate of the birthweight equation in which higher-order terms and interactions have been added can be seen in the third column of Table~\ref{reg:ols}. Though this method may produce more accurate estimates, it can be less precise and is computationally more intensive.

\begin{table}[htbp]\centering
\def\sym#1{\ifmmode^{#1}\else\(^{#1}\)\fi}
\caption{First-Difference Estimates of the Effect of TSPs Pollution on Log Housing Values\label{reg:fd}}
\begin{tabular}{l*{3}{D{.}{.}{-1}}}
\toprule
                    &\multicolumn{1}{c}{(1)}         &\multicolumn{1}{c}{(2)}         &\multicolumn{1}{c}{(3)}         \\
\midrule
Mean TSPs ($\times 1/100$)&       0.100\sym{***}&       0.023         &      -0.003         \\
                    &     (0.031)         &     (0.019)         &     (0.016)         \\
Controls            &\multicolumn{1}{c}{\ensuremath{\text{No}}}         &\multicolumn{1}{c}{\ensuremath{\text{Yes}}}         &\multicolumn{1}{c}{\ensuremath{\text{Yes}}}         \\
\midrule
\ensuremath{R^2}    &       0.017         &       0.553         &       0.854         \\
\ensuremath{F}      &      10.083         &      40.201         &           .         \\
\ensuremath{N}      &\multicolumn{1}{c}{1000}         &\multicolumn{1}{c}{995}         &\multicolumn{1}{c}{995}         \\
\bottomrule
\multicolumn{4}{l}{\footnotesize Equation (3) includes quadratics, cubics, and interactions of the controls.}\\
\multicolumn{4}{l}{\footnotesize Huber-White standard errors in parentheses.}\\
\multicolumn{4}{l}{\footnotesize \sym{*} \(p<0.10\), \sym{**} \(p<0.05\), \sym{***} \(p<0.01\)}\\
\end{tabular}
\end{table}


\item The propensity score approach assigns a probability of treatment, conditional on the observables, to each individual. If the treatment is randomly assigned conditional on the observables, then the propensity score is a sufficient statistic for selection bias. For this reason, it can be used to correct the unadjusted estimates. A benefit of this approach is that it only requires a single index to adjust estimates, while that of (c) requires many regressors.

\item A propensity score was estimated for each individual as follows: (i) a logit model containing interactions and higher order terms of the the observables was used to estimate propensity scores; (ii) the sample was stratified into blocks in which p-scores were not significantly different from each other at the 1\% level; (iii) the observables in each block were tested for significant differences in their sample means; and (iv) if 90\% of the observables in 90\% of the blocks were statistically indistinguishable at the 1\% level, the algorithm terminated. This produced the 29 blocks that are summarized in Table~\ref{tab:blocks}. Note that there are treatments and controls in every block.

\begin{table}[htbp]\centering
\def\sym#1{\ifmmode^{#1}\else\(^{#1}\)\fi}
\caption{Mean Comparison Tests\label{tab:meant}}
\begin{tabular}{l*{3}{D{.}{.}{-1}}}
\toprule
                    &\multicolumn{2}{c}{Difference}    \\
\midrule
Hourly Wage         &      -6.670\sym{***}&     (-5.48)\\
Age                 &       6.073\sym{***}&      (5.00)\\
Female              &      0.0425         &      (0.81)\\
White               &      0.0451         &      (1.42)\\
Self-Employed       &     -0.0151         &     (-0.44)\\
Union Member        &       0.122\sym{**} &      (2.62)\\
Ever Married        &       0.342\sym{***}&      (7.30)\\
Father's Educ.      &      -2.425\sym{***}&     (-8.12)\\
Mother's Educ.      &      -1.565\sym{***}&     (-6.12)\\
\midrule
\ensuremath{N}      &\multicolumn{1}{c}{362}         &            \\
\bottomrule
\multicolumn{3}{l}{\footnotesize High School minus College mean values;}\\
\multicolumn{3}{l}{\footnotesize \ensuremath{t}-statistics in parentheses.}\\
\multicolumn{3}{l}{\footnotesize \sym{*} \(p<0.05\), \sym{**} \(p<0.01\), \sym{***} \(p<0.001\)}\\
\end{tabular}
\end{table}


A box plot of propensity scores for non-smokers and smokers can be seen in Figure~\ref{fig:boxplot}. Evidently there is overlap in the treatments and controls, though there is selection on the observables.

\begin{figure}[htbp!]
\centering
\includegraphics[width=0.70\textwidth]{../Figures/fig1.eps}
\caption{Propensity Score Distribution of Non-smokers and Smokers}
\label{fig:boxplot}
\end{figure}

An estimate of the average treatment effect, adjusted using the p-score, can be seen in the first column of Table~\ref{reg:wls}. If the average effect of treatment on the treated is calculated by taking a weighted average of differences in group means across the 29 blocks, this produces a statistic of $-207.792$. This statistic is slightly smaller than the p-score-adjusted estimate, but it is not significantly different.

\item Estimates of the average treatment effect and average effect of treatment on the treated can be seen in the second and third columns of Table~\ref{reg:wls}, respectively. Note that these estimates are not significantly different from those of (e). This suggests that the p-score adjustment is robust to selection bias. A benefit of using the p-scores as weights is that it produces more efficient estimates. However, it puts greater weight on treatments with low propensity scores and controls with high propensity scores, so can bias estimates by putting a lot of weight on a small number of observations that may have measurement error or imprecisely estimated p-scores.

A plot of average estimated p-scores versus the actual fraction of smokers in blocks of 200 can be seen in Figure~\ref{fig:45deg}. Note that the p-scores ``track" the 45-degree line, suggesting that the weighting procedure is reasonable.

\begin{figure}[htbp!]
\centering
\includegraphics[width=0.85\textwidth]{../Figures/fig2.eps}
\caption{Propensity Score vs. Fraction of Smokers}
\label{fig:45deg}
\end{figure}

The fourth column of Table~\ref{reg:wls} contains an estimate of the average treatment effect on the treated, using the fraction of smokers in each of the 200 blocks to weight the observations. Note that this estimate is not significantly different from those obtained using the estimated p-scores as weights, providing further evidence of the reasonableness of those estimates.

\begin{table}[htbp]\centering \footnotesize
\def\sym#1{\ifmmode^{#1}\else\(^{#1}\)\fi}
\caption{2SLS Results for Average Gain Scores, Using Eligibility for P-900 as an Instrument\label{reg:2SLS}}
\begin{tabular}{l*{6}{D{.}{.}{-1}}}
\toprule
                    &\multicolumn{3}{c}{1988--1990}                                   &\multicolumn{3}{c}{1988--1992}                                   \\\cmidrule(lr){2-4}\cmidrule(lr){5-7}
                    &\multicolumn{1}{c}{(1)}         &\multicolumn{1}{c}{(2)}         &\multicolumn{1}{c}{(3)}         &\multicolumn{1}{c}{(4)}         &\multicolumn{1}{c}{(5)}         &\multicolumn{1}{c}{(6)}         \\
\midrule
\multicolumn{7}{l}{\em Panel A: First-stage estimates (P-900)} \\
Eligible            &       0.872\sym{***}&       0.842\sym{***}&       0.841\sym{***}&       0.872\sym{***}&       0.842\sym{***}&       0.841\sym{***}\\
                    &     (0.034)         &     (0.068)         &     (0.068)         &     (0.034)         &     (0.068)         &     (0.068)         \\
\ensuremath{\overline{y_{j}^{88}}}&                     &       0.025         &       0.018         &                     &       0.025         &       0.018         \\
                    &                     &     (0.087)         &     (0.086)         &                     &     (0.087)         &     (0.086)         \\
\ensuremath{\left(\overline{y_{j}^{88}}\right)^{2}}&                     &      -0.001         &      -0.000         &                     &      -0.001         &      -0.000         \\
                    &                     &     (0.001)         &     (0.001)         &                     &     (0.001)         &     (0.001)         \\
\ensuremath{\left(\overline{y_{j}^{88}}\right)^{3}}&                     &       0.000         &       0.000         &                     &       0.000         &       0.000         \\
                    &                     &     (0.000)         &     (0.000)         &                     &     (0.000)         &     (0.000)         \\
\ensuremath{N_{j}^{88}}&                     &                     &       0.001         &                     &                     &       0.001         \\
                    &                     &                     &     (0.001)         &                     &                     &     (0.001)         \\
\ensuremath{\left(N_{j}^{88}\right)^{2}}&                     &                     &      -0.000         &                     &                     &      -0.000         \\
                    &                     &                     &     (0.000)         &                     &                     &     (0.000)         \\
\ensuremath{\overline{y_{j}^{88}}\cdot N_{j}^{88}}&                     &                     &      -0.000         &                     &                     &      -0.000         \\
                    &                     &                     &     (0.000)         &                     &                     &     (0.000)         \\
\ensuremath{R^{2}}  &       0.804         &       0.806         &       0.806         &       0.804         &       0.806         &       0.806         \\
\addlinespace
\multicolumn{7}{l}{\em Panel B: Reduced-form estimates (Average gain scores)} \\
Eligible            &       4.975\sym{***}&       1.333         &       1.317         &       7.871\sym{***}&       3.453\sym{**} &       3.420\sym{**} \\
                    &     (0.825)         &     (1.450)         &     (1.452)         &     (0.830)         &     (1.457)         &     (1.465)         \\
\ensuremath{\overline{y_{j}^{88}}}&                     &      -0.415         &      -0.643         &                     &      -1.488         &      -1.866         \\
                    &                     &     (1.515)         &     (1.551)         &                     &     (1.592)         &     (1.608)         \\
\ensuremath{\left(\overline{y_{j}^{88}}\right)^{2}}&                     &       0.004         &       0.008         &                     &       0.022         &       0.027         \\
                    &                     &     (0.026)         &     (0.027)         &                     &     (0.027)         &     (0.028)         \\
\ensuremath{\left(\overline{y_{j}^{88}}\right)^{3}}&                     &      -0.000         &      -0.000         &                     &      -0.000         &      -0.000         \\
                    &                     &     (0.000)         &     (0.000)         &                     &     (0.000)         &     (0.000)         \\
\ensuremath{N_{j}^{88}}&                     &                     &       0.031         &                     &                     &       0.042         \\
                    &                     &                     &     (0.045)         &                     &                     &     (0.039)         \\
\ensuremath{\left(N_{j}^{88}\right)^{2}}&                     &                     &      -0.000         &                     &                     &      -0.000\sym{**} \\
                    &                     &                     &     (0.000)         &                     &                     &     (0.000)         \\
\ensuremath{\overline{y_{j}^{88}}\cdot N_{j}^{88}}&                     &                     &      -0.000         &                     &                     &       0.000         \\
                    &                     &                     &     (0.001)         &                     &                     &     (0.001)         \\
\ensuremath{R^{2}}  &       0.056         &       0.116         &       0.118         &       0.131         &       0.188         &       0.201         \\
\addlinespace
\multicolumn{7}{l}{\em Panel C: IV estimates (Average gain scores)} \\
P-900               &       5.708\sym{***}&       1.583         &       1.567         &       9.030\sym{***}&       4.102\sym{**} &       4.068\sym{**} \\
                    &     (0.946)         &     (1.723)         &     (1.727)         &     (0.953)         &     (1.731)         &     (1.743)         \\
\ensuremath{\overline{y_{j}^{88}}}&                     &      -0.455         &      -0.671         &                     &      -1.592         &      -1.939         \\
                    &                     &     (1.493)         &     (1.535)         &                     &     (1.567)         &     (1.591)         \\
\ensuremath{\left(\overline{y_{j}^{88}}\right)^{2}}&                     &       0.005         &       0.009         &                     &       0.024         &       0.029         \\
                    &                     &     (0.026)         &     (0.027)         &                     &     (0.027)         &     (0.027)         \\
\ensuremath{\left(\overline{y_{j}^{88}}\right)^{3}}&                     &      -0.000         &      -0.000         &                     &      -0.000         &      -0.000         \\
                    &                     &     (0.000)         &     (0.000)         &                     &     (0.000)         &     (0.000)         \\
\ensuremath{N_{j}^{88}}&                     &                     &       0.029         &                     &                     &       0.036         \\
                    &                     &                     &     (0.045)         &                     &                     &     (0.039)         \\
\ensuremath{\left(N_{j}^{88}\right)^{2}}&                     &                     &      -0.000         &                     &                     &      -0.000\sym{**} \\
                    &                     &                     &     (0.000)         &                     &                     &     (0.000)         \\
\ensuremath{\overline{y_{j}^{88}}\cdot N_{j}^{88}}&                     &                     &      -0.000         &                     &                     &       0.000         \\
                    &                     &                     &     (0.001)         &                     &                     &     (0.001)         \\
\ensuremath{R^{2}}  &       0.056         &       0.116         &       0.118         &       0.131         &       0.188         &       0.201         \\
\addlinespace
\midrule
\ensuremath{N}      &\multicolumn{1}{c}{658}         &\multicolumn{1}{c}{658}         &\multicolumn{1}{c}{658}         &\multicolumn{1}{c}{651}         &\multicolumn{1}{c}{651}         &\multicolumn{1}{c}{651}         \\
\bottomrule
\multicolumn{7}{l}{\tiny Huber-White standard errors in parentheses.}\\
\multicolumn{7}{l}{\tiny \sym{*} \(p<0.10\), \sym{**} \(p<0.05\), \sym{***} \(p<0.01\)}\\
\end{tabular}
\end{table}


\item A plot of average birthweight versus average estimated p-scores in blocks of 100, for non-smokers and smokers separately, can be seen in Figure~\ref{fig:block_bw1}. An analogous plot that combines non-smokers and smokers can be seen in Figure~\ref{fig:block_bw2}. These figures show that mean birthweight declines with propensity to smoke, regardless of smoking status, and that it is systematically lower (by about 200g) for smokers than for non-smokers. This is consistent with the bias in the unadjusted estimates (note especially the steeper slope in Figure~\ref{fig:block_bw2}), and with the robustness of the p-score-adjusted and weighted estimates.

\begin{figure}[htbp!]
\centering
\includegraphics[width=0.85\textwidth]{../Figures/fig3.eps}
\caption{Mean Birthweight vs. Propensity Score}
\label{fig:block_bw1}
\end{figure}

\begin{figure}[htbp!]
\centering
\includegraphics[width=0.85\textwidth]{../Figures/fig4.eps}
\caption{Mean Birthweight vs. Propensity Score}
\label{fig:block_bw2}
\end{figure}

\begin{figure}[htbp!]
\centering
\includegraphics[width=0.85\textwidth]{../Figures/fig5.eps}
\caption{Low Birthweight vs. Propensity Score}
\label{fig:block_lbw1}
\end{figure}

\begin{figure}[htbp!]
\centering
\includegraphics[width=0.85\textwidth]{../Figures/fig6.eps}
\caption{Low Birthweight vs. Propensity Score}
\label{fig:block_lbw2}
\end{figure}

\item Weighted estimates of the average treatment effect of smoking on the probability of low birthweight can be seen in Table~\ref{reg:wlslbw}. A plot of low birthweight versus average estimated p-scores in blocks of 100, for non-smokers and smokers separately, can be seen in Figure~\ref{fig:block_lbw1}. An analogous plot that combines non-smokers and smokers can be seen in Figure~\ref{fig:block_lbw2}. These figures show that the probability of low birthweight increases with propensity to smoke, regardless of smoking status, and that it is systematically (though marginally) higher for smokers than for non-smokers.

\begin{table}[htbp]\centering
\def\sym#1{\ifmmode^{#1}\else\(^{#1}\)\fi}
\caption{2SLS Results for 1988{--}1992 Average Gain Scores, within Narrow Bands of the Selection Threshold\label{reg:2SLSbw}}
\begin{tabular}{l*{3}{D{.}{.}{-1}}}
\toprule
                    &\multicolumn{1}{c}{(1)}&\multicolumn{1}{c}{(2)}&\multicolumn{1}{c}{(3)}\\
                    &\multicolumn{1}{c}{Full Sample}&\multicolumn{1}{c}{\ensuremath{\pm7} Points}&\multicolumn{1}{c}{\ensuremath{\pm3} Points}\\
\midrule
P-900               &       4.068\sym{**} &       4.168         &       6.916\sym{*}  \\
                    &     (1.743)         &     (2.840)         &     (3.693)         \\
\ensuremath{\overline{y_{j}^{88}}}&      -1.939         &      66.462         &     251.391         \\
                    &     (1.591)         &    (64.042)         &   (883.807)         \\
\ensuremath{\left(\overline{y_{j}^{88}}\right)^{2}}&       0.029         &      -1.510         &      -5.426         \\
                    &     (0.027)         &     (1.465)         &    (20.385)         \\
\ensuremath{\left(\overline{y_{j}^{88}}\right)^{3}}&      -0.000         &       0.011         &       0.039         \\
                    &     (0.000)         &     (0.011)         &     (0.156)         \\
\ensuremath{N_{j}^{88}}&       0.036         &       0.113         &      -0.778         \\
                    &     (0.039)         &     (0.157)         &     (0.587)         \\
\ensuremath{\left(N_{j}^{88}\right)^{2}}&      -0.000\sym{**} &      -0.000         &       0.001\sym{*}  \\
                    &     (0.000)         &     (0.000)         &     (0.001)         \\
\ensuremath{\overline{y_{j}^{88}}\cdot N_{j}^{88}}&       0.000         &      -0.002         &       0.013         \\
                    &     (0.001)         &     (0.004)         &     (0.012)         \\
\midrule
\ensuremath{R^2}    &       0.201         &       0.113         &       0.106         \\
\ensuremath{F}      &      30.553         &       4.672         &           .         \\
\ensuremath{N}      &\multicolumn{1}{c}{651}         &\multicolumn{1}{c}{245}         &\multicolumn{1}{c}{102}         \\
\bottomrule
\multicolumn{4}{l}{\footnotesize Huber-White standard errors in parentheses.}\\
\multicolumn{4}{l}{\footnotesize \sym{*} \(p<0.10\), \sym{**} \(p<0.05\), \sym{***} \(p<0.01\)}\\
\end{tabular}
\end{table}


\item Estimates of the effect of smoking on infant death can be seen in Table~\ref{reg:death}. Note that once observables are controlled for, the effect diminishes by an order of magnitude and is no longer significant. Neither is it significant when observations are weighted using their estimated p-scores, as in columns (3) and (4). This suggests that infant deaths are not caused by smoking-related reductions in birthweight, but instead caused by other characteristics that also make a woman more likely to smoke.

\begin{table}[htbp]\centering
\def\sym#1{\ifmmode^{#1}\else\(^{#1}\)\fi}
\caption{OLS, 2SLS, and \ensuremath{1^{\textrm{st}}}-Diff. Estimates of Log Wage Equations\label{reg:ols_2SLS_Diff}}
\begin{tabular}{l*{4}{D{.}{.}{-1}}}
\toprule
            &\multicolumn{1}{c}{(1)}&\multicolumn{1}{c}{(2)}&\multicolumn{1}{c}{(3)}&\multicolumn{1}{c}{(4)}\\
            &\multicolumn{1}{c}{OLS}&\multicolumn{1}{c}{2SLS}&\multicolumn{1}{c}{\ensuremath{1^{\textrm{st}}}-Diff.}&\multicolumn{1}{c}{\ensuremath{1^{\textrm{st}}}-Diff. 2SLS}\\
\midrule
Constant    &      -1.095\sym{***}&      -1.188\sym{***}&                     &                     \\
            &     (0.261)         &     (0.269)         &                     &                     \\
            &     [0.292]         &     [0.306]         &                     &                     \\
            &   \{0.339\}         &   \{0.354\}         &                     &                     \\
Education   &       0.110\sym{***}&       0.116\sym{***}&       0.062\sym{**} &       0.108\sym{***}\\
            &     (0.010)         &     (0.010)         &     (0.019)         &     (0.030)         \\
            &     [0.010]         &     [0.011]         &     [0.020]         &     [0.034]         \\
            &   \{0.013\}         &   \{0.014\}         &                     &                     \\
Age         &       0.104\sym{***}&       0.104\sym{***}&                     &                     \\
            &     (0.010)         &     (0.011)         &                     &                     \\
            &     [0.012]         &     [0.012]         &                     &                     \\
            &   \{0.015\}         &   \{0.015\}         &                     &                     \\
\ensuremath{\text{Age}^{2}}&      -0.001\sym{***}&      -0.001\sym{***}&                     &                     \\
            &     (0.000)         &     (0.000)         &                     &                     \\
            &     [0.000]         &     [0.000]         &                     &                     \\
            &   \{0.000\}         &   \{0.000\}         &                     &                     \\
Female      &      -0.318\sym{***}&      -0.316\sym{***}&                     &                     \\
            &     (0.040)         &     (0.040)         &                     &                     \\
            &     [0.040]         &     [0.040]         &                     &                     \\
            &   \{0.049\}         &   \{0.049\}         &                     &                     \\
White       &      -0.100         &      -0.098         &                     &                     \\
            &     (0.072)         &     (0.072)         &                     &                     \\
            &     [0.068]         &     [0.068]         &                     &                     \\
            &   \{0.069\}         &   \{0.068\}         &                     &                     \\
\midrule
\ensuremath{R^2}&       0.339         &       0.338         &       0.031         &           .         \\
\ensuremath{F}&      69.058         &      67.581         &      10.880         &           .         \\
\ensuremath{N}&\multicolumn{1}{c}{680}         &\multicolumn{1}{c}{680}         &\multicolumn{1}{c}{340}         &\multicolumn{1}{c}{340}         \\
\bottomrule
\multicolumn{5}{l}{\footnotesize Own-reports of education instrumented for with twin-reports.}\\
\multicolumn{5}{l}{\footnotesize Standard errors in parentheses; robust standard errors in brackets;}\\
\multicolumn{5}{l}{\footnotesize clustered standard errors in braces.}\\
\multicolumn{5}{l}{\footnotesize \sym{*} \(p<0.05\), \sym{**} \(p<0.01\), \sym{***} \(p<0.001\)}\\
\end{tabular}
\end{table}


\item Sample sizes and smoking rates versus age can be seen in Figure~\ref{fig:lifec}. Average birthweight versus age can be seen in Figure~\ref{fig:birwt}. Evidently younger women are more likely to smoke. Moreover, younger women (and smokers, regardless of age) are more likely to have low birthweight infants.

\item In summary, these results suggest that smoking reduces birthweight by roughly 210g, and that smoking does not affect infant deaths. Mothers who smoke are already more likely to have low birthweight infants, regardless of their smoking status. It is likely this propensity for low birthweight that is driving infant mortality among smoking mothers. The robustness of the estimated average treatment effect of maternal smoking to alternative estimation methods indicates its credibility. 

\begin{figure}[htbp!]
\centering
\includegraphics[width=0.85\textwidth]{../Figures/fig7.eps}
\caption{Sample Size and Smoking Rate vs. Age}
\label{fig:lifec}
\end{figure}

\begin{figure}[htbp!]
\centering
\includegraphics[width=0.85\textwidth]{../Figures/fig8.eps}
\caption{Average Birthweight vs. Age}
\label{fig:birwt}
\end{figure}

\end{enumerate}
\end{enumerate}

\bigskip

\appendix

\section{STATA Source Code}
\begin{verbatim}
cd "E:\2011 Fall\Labor and Population Economics\Problem Sets\PS2"

local nblock = 5
local level = 0.01

local fcount = 0

set more off

set matsize 11000

cap log close PS2
log using PS2, replace text name(PS2)

cap which estout
if _rc ssc install estout
cap which savesome
if _rc ssc install savesome

set scheme s1mono
graph set eps logo off
graph set eps preview off
graph set eps orientation portrait
graph set eps mag 100

use smoking2.dta, replace
// outsheet using smoking2.txt, nolabel replace

la define tobla 0 "Non-smokers" 1 "Smokers"
la values tobacco tobla

rename mblack	dmblack
rename mhispan	dmhispan
rename motherr	dmotherr

drop if dmblack+dmhispan+dmotherr>1

gen dmwhite		= ~(dmblack | dmhispan | dmotherr)
gen dfwhite		= ~(fblack | fhispan | fotherr)
gen dmage2		= dmage^2
gen dmage3		= dmage^3
gen dmage4		= dmage^4
gen dmeduc2		= dmeduc^2
gen disllb2		= disllb^2


la var dmage	"Mother Age"
la var dmage2	"Mother Age$^2$"
la var dmeduc	"Mother Education"
la var dmeduc2	"Mother Education$^2$"
la var dmar		"Mother Unmarried"
la var dfage	"Father Age"
la var dfeduc	"Father Education"
la var dfwhite	"Father White"
la var alcohol	"Mother Drank"
la var nprevist	"Prenatal Visits"
la var deadkids	"Previous Deaths"
la var diabete	"Mother Diabetic"
la var anemia	"Mother Anemic"
la var tobacco	"Mother Smoked"
la var dmwhite	"Mother White"
la var dmblack	"Mother Black"
la var dmhispan	"Mother Hispanic"
la var dmotherr	"Mother Other Race"
la var drink	"Num. Drinks"
la var dlivord	"Birth Order"
la var disllb	"Mo. Since Last"
la var disllb2	"Mo. Since Last$^2$"
la var preterm	"Prev. Birth Pre-Term"
la var pre4000	"Prev. Birth $>4\text{kg}$"
la var plural	"Twins or Greater"
la var phyper	"Hypertension"

eststo r_bw_t: reg dbirwt tobacco, vce(robust)

local mtvars dmage dmeduc dmwhite dmar dfage dfeduc dfwhite alcohol nprevist ///
	deadkids diabete anemia
qui eststo mt_not: estpost summarize `mtvars' if tobacco==0
qui eststo mt_tob: estpost summarize `mtvars' if tobacco==1
qui eststo mt_ttl: estpost summarize `mtvars' if tobacco~=.
qui eststo tt_tob: estpost ttest `mtvars', by(tobacco)
qui estadd mat mean=e(b)
qui estadd mat sd=e(se)

esttab mt_not mt_tob mt_ttl tt_tob using TEXDocs/tab1.tex, replace ///
	cells(mean(fmt(%9.3f) star) sd(fmt(%9.3f) par)) ///
	stats(N, fmt(%9.0g) layout("\multicolumn{1}{c}{@}") ///
		labels("\ensuremath{N}")) ///
	nodepvar ///
	nonumber ///
	collabels(none) ///
	mtitles("Non-Smokers" "Smokers" "Total" "Difference") ///
	label ///
	legend ///
	note("Mean values; standard errors in parentheses.") ///
	title("Descriptive Statistics\label{tab:groupm}") ///
	substitute(none) ///
	gaps ///
	alignment(D{.}{.}{-1}) ///
	booktabs ///
	style(tex)

eststo r_bw_tc: reg dbirwt tobacco dmage dmage2 dmeduc dmeduc2 dmar ///
	dmblack dmhispan dmotherr alcohol nprevist disllb preterm pre4000 ///
	plural phyper diabete, vce(robust)

egen category = group(dmar dmblack dmhispan dmotherr)
egen dmrace = group(dmblack dmhispan dmotherr)
xi: eststo r_bw_tmvm: reg dbirwt ///
	tobacco ///
	alcohol ///
	i.category ///
	i.dmrace|preterm ///
	i.category|pre4000 ///
	i.category|plural ///
	i.category|phyper ///
	i.category|diabete ///
	///
	i.category|dmage ///
	i.category|dmage2 ///
	i.category|dmage3 ///
	i.category|dmage4 ///
	i.category|dmeduc ///
	i.category|dmeduc2 ///
	i.category|nprevist ///
	i.category|disllb ///
	i.category|disllb2, vce(robust)

// Regression Table for:
//   dbirwt on tobacco 
//   dbirwt on tobacco and covariates
//   dbirwt on tobacco and covariates, interactions, and 2nd order terms
esttab r_bw_t r_bw_tc r_bw_tmvm using TEXDocs/reg1.tex, replace ///
	order(_cons tobacco dmage dmage2 dmage3 dmage4 dmeduc dmeduc2 dmar dmblack ///
		dmhispan dmotherr alcohol nprevist disllb disllb2 preterm pre4000 ///
		plural phyper diabete) ///
	varlabels(_cons "Constant") ///
	cells(b(star fmt(%9.3f)) se(par fmt(%9.3f))) ///
	stats(r2 F N, fmt(%9.3f %9.3f %9.0g) ///
		layout("@" "@" "\multicolumn{1}{c}{@}") ///
		labels("\ensuremath{R^2}" "\ensuremath{F}" "\ensuremath{N}")) ///
	indicate("Int. and $2^{\text{nd}}$ Order"= _I* *3 *4 disllb2, ///
		labels("\multicolumn{1}{c}{\ensuremath{\text{Yes}}}" ///
		"\multicolumn{1}{c}{\ensuremath{\text{No}}}")) ///
	nomtitles ///
	label ///
	collabels(none) ///
	legend ///
	addnotes("Dependent variable is infant birthweight." ///
		"Robust standard errors in parentheses.") ///
	title("OLS Estimates of Birthweight Equations\label{reg:ols}") ///
	alignment(D{.}{.}{-1}) ///
	booktabs ///
	style(tex)

// proceed with pscore estimation 
capture drop pscore
capture drop block
capture drop block_inf
capture drop block_sup
// estimate the propensity score
xi: logit tobacco ///
	i.category ///
	alcohol ///
	i.dmrace|preterm ///
	i.category|pre4000 ///
	i.category|plural ///
	i.category|phyper ///
	i.category|diabete ///
	///
	i.category|dmage ///
	i.category|dmage2 ///
	i.category|dmage3 ///
	i.category|dmage4 ///
	i.category|dmeduc ///
	i.category|dmeduc2 ///
	i.category|nprevist ///
	i.category|disllb ///
	i.category|disllb2 ///

predict double pscore
la var pscore "Propensity Score"
sum pscore, detail

// divide the data into blocks
gen id = _n // store obs. ids so we can restore the original order later
xtile block = pscore, nq(`nblock')
by block, sort: egen block_inf = min(pscore)
by block, sort: egen block_sup = max(pscore)
la var block "Block No."
la var block_inf "Block Infimum"
la var block_sup "Block Supremum"

// subdivide blocks until pscores are balanced
tab block tobacco
local iblock = 1
while `iblock'<=`nblock' {
	di "block `iblock' of `nblock'..."
	// count the number of treated/contols in this block
	qui count if block == `iblock' & tobacco==0
	local ncontrol = r(N)
	qui count if block == `iblock' & tobacco==1
	local ntreated = r(N)
	// skip blocks with no treatments or no controls in them
	if `ncontrol' == 0 | `ntreated' == 0 {
		local iblock = `iblock'+1
	}
	else {
		local ivar = 1
		local nvar = 0
		local fail_count = 0
		// test for difference in mean values
		qui ttest pscore if block==`iblock', by(tobacco)
		if r(p)<`level' {
			local fail_count = `fail_count'+1
		}
		// if pscores are different, split the block
		if `fail_count'>0 {
			// shift the remaining blocks up
			qui replace block = block+1 if block>`iblock' & block!=.
			local nblock = `nblock'+1
			// get the block split point
			sum block_inf if block==`iblock', meanonly
			local temp_inf = r(mean)
			sum block_sup if block==`iblock', meanonly
			local temp_sup = r(mean)
			local split = (`temp_inf' + `temp_sup')/2
			// split the block
			di "splitting block `iblock'"
			qui replace block = block+1 if block==`iblock' & pscore>=`split' ///
				& pscore<=block_sup
			qui replace block_sup = `split' if block==`iblock'
			qui replace block_inf = `split' if block==`iblock'+1
		} // end if
		else {
			// pscores are not different, iterate to the next block
			local iblock = `iblock'+1
		} // end else
	} // end else
} // end while
tab block tobacco

// check that the covariates are balanced
local total_fail = 0
local iblock = 1
while `iblock'<=`nblock' {
	di "block `iblock' of `nblock'..."
	// count the number of treated/contols in this block
	qui count if block == `iblock' & tobacco==0
	local ncontrol = r(N)
	qui count if block == `iblock' & tobacco==1
	local ntreated = r(N)
	// skip blocks with no treatments or no controls in them
	if `ncontrol' == 0 | `ntreated' == 0 {
		local iblock = `iblock'+1
	}
	else {
		local ivar = 1
		local nvar = 0
		local fail_count = 0
		// test for difference in mean values
		foreach var in tobacco dmage dmeduc dmar dmblack dmhispan ///
			dmotherr alcohol nprevist disllb preterm ///
			pre4000 plural phyper diabete {
			qui ttest `var' if block==`iblock', by(tobacco)
			if r(p)<`level' {
				di "`var' fail in block `iblock'!"
				local fail_count = `fail_count'+1
			}
			local nvar = `nvar'+1
		} // end foreach
		// if 10% or more mean values are different, fail
		if (`fail_count')/`nvar'>=.10 {
			local total_fail = `total_fail'+1
		} // end if
		// iterate to the next block
		local iblock = `iblock'+1
	} // end else
} // end while
di "`total_fail' (" round(`total_fail'/`nblock'*100) ///
		"%) blocks were unbalanced"
// append the pscore data to the file on disk
preserve
sort id
keep pscore block block_inf block_sup
merge 1:1 _n using smoking2.dta, nogen
save smoking2.dta, replace
restore

// graph the pscore boxplot
local fcount = `fcount'+1
graph box pscore, over(tobacco) ///
	title(Propensity Score Box Plot) legend(off) ///
	name(fig`fcount', replace) saving(Figures\fig`fcount'.gph, replace)
graph export Figures\fig`fcount'.eps, replace

// tabulate the blocks
estpost tab block tobacco
esttab . using TEXDocs/tab2.tex, replace ///
	cell(b(fmt(g)) count(fmt(g) par keep(Total))) ///
	collabels(none) unstack noobs nonumber nomtitle  ///
	eqlabels(, lhs("Block No.")) ///
	varlabels(, blist(Total "{hline @width}{break}")) ///
	title("Propensity Score Blocks\label{tab:blocks}") ///
	alignment(D{.}{.}{-1}) ///
	booktabs ///
	style(tex)

// regression-adjust using the pscore
local pscore_mean = sum(pscore)/sum(pscore~=.)
disp `pscore_mean'
gen tobacco_pscore = tobacco*(pscore-`pscore_mean')
la var tobacco_pscore "\ensuremath{\text{Mother Smoked}\cdot\text{P-Score}}"
eststo r_bw_tp: reg dbirwt tobacco pscore tobacco_pscore, vce(robust)

// stratify the sample and estimate TOT
drop if pscore == .
// calculate the total number of treatments
egen nT = total(tobacco)
// claulate number of treatments and controls in each block
egen block_nT = total(tobacco), by(block)
egen block_nC = total(~tobacco), by(block)
// not calculate TOT from the stratified sample
gen block_unwtTOT = tobacco*dbirwt/block_nT-(~tobacco)*dbirwt/block_nC
gen block_wt = block_nT/nT
preserve
collapse (sum) block_unwtTOT (first) block_wt, by(block)
gen TOT = block_unwtTOT*block_wt
collapse (sum) TOT
disp "TOT=" TOT
restore

// low birthweight
gen lowbw = (dbirwt<2500)
// ATE and TOT probability weights
gen ate_weight = (~tobacco)*1/(1-pscore)+tobacco*1/pscore
gen tot_weight = (~tobacco)*pscore/(1-pscore)+tobacco

foreach var in dbirwt lowbw {
	if "`var'"=="dbirwt" {
		local yvar = "Mean birth weight, by cell"
		local gtitle = "Mean Birthweight vs. Propensity Score"
	}
	else {
		local yvar = "Fraction low birthweight, by cell"
		local gtitle = "Fraction Low Birthweight vs. Propensity Score"
	}

	// regress using the pscores as weights
	eststo r_bw_tpw_ate_`var': reg `var' tobacco [pw=ate_weight], vce(robust)
	eststo r_bw_tpw_tot_`var': reg `var' tobacco [pw=tot_weight], vce(robust)

	if "`var'"=="dbirwt" {
		// check reasonableness of reweighing procedure
		capture drop block200
		capture drop block200_nT
		capture drop block200_nC
		capture drop block200_fracT
		xtile block200 = pscore, nq(200)
		egen block200_nT = total(tobacco), by(block200)
		egen block200_nC = total(~tobacco), by(block200)
		gen block200_fracT = block200_nT/(block200_nT+block200_nC)
		preserve
		collapse (mean) pscore (first) block200_fracT, by(block200)
		local fcount = `fcount'+1
		twoway scatter pscore block200_fracT || line block200_fracT block200_fracT || , ///
			title(Propensity Score vs. Actual Fraction Smoker) ///
			xtitle("Actual fraction smoker, by cell") ///
			ytitle("Estimated p-score, by cell") ///
			legend(on label(1 "Estimated p-score") label(2 "45-degree line")) ///
			name(fig`fcount', replace) saving(Figures\fig`fcount'.gph, replace)
		graph export Figures\fig`fcount'.eps, replace
		restore
		// estimate TOT using actual fraction of smokers in each block
		capture drop tot_weight200
		gen tot_weight200 = (~tobacco)*block200_fracT/(1-block200_fracT)+tobacco
		eststo r_bw_tpw_tot200_`var': reg `var' tobacco [pw=tot_weight200], vce(robust)
	}

	// compare "non-parametric" means
	preserve
	keep if tobacco==1
	capture drop block100
	xtile block100 = pscore, nq(100)
	collapse (mean) `var' pscore (first) tobacco, by(block100)
	save smokers, replace
	restore
	preserve
	keep if tobacco==0
	capture drop block100
	xtile block100 = pscore, nq(100)
	collapse (mean) `var' pscore (first) tobacco, by(block100)
	append using smokers
	local fcount = `fcount'+1
	twoway scatter `var' pscore if tobacco==0 || scatter `var' pscore if tobacco==1 || , ///
		title("`gtitle'") ///
		xtitle("Estimated p-score, by cell") ///
		ytitle("`yvar'") ///
		legend(on label(1 "Non-smokers") label(2 "Smokers")) ///
		name(fig`fcount', replace) saving(Figures\fig`fcount'.gph, replace)
	graph export Figures\fig`fcount'.eps, replace
	erase smokers.dta
	restore
	preserve
	capture drop block200
	xtile block200 = pscore, nq(200)
	collapse (mean) `var' pscore (first) tobacco, by(block200)
	local fcount = `fcount'+1
	scatter `var' pscore, ///
		title("`gtitle'") ///
		xtitle("Estimated p-score, by cell") ///
		ytitle("`yvar'") ///
		name(fig`fcount', replace) saving(Figures\fig`fcount'.gph, replace)
	graph export Figures\fig`fcount'.eps, replace
	restore
} // end for

// effect of smoking on infant death
eststo r_d_t: reg death tobacco, vce(robust)
eststo r_d_tc: reg death tobacco dmage dmage2 dmeduc dmeduc2 dmar ///
	dmblack dmhispan dmotherr alcohol nprevist disllb preterm pre4000 ///
	plural phyper diabete, vce(robust)
eststo r_d_tpw_ate: reg death tobacco [pw=ate_weight], vce(robust)
eststo r_d_tpw_tot: reg death tobacco [pw=tot_weight], vce(robust)

// smoking lifecycle
preserve
egen ageblock = group(dmage)
collapse (sum) ageblock_nT=tobacco ageblock_nC=~tobacco (first) dmage ///
	(count) ageblock_n=dmage, by(ageblock)
gen ageblock_fracT = ageblock_nT/ageblock_n
local fcount = `fcount'+1
twoway scatter ageblock_n dmage || scatter ageblock_fracT dmage, yaxis(2) || , ///
		title("Sample Size and Smoking Rate vs. Age") ///
		xtitle("Age") ///
		ytitle("Sample Size", axis(1)) ///
		ytitle("Smoking Rate", axis(2)) ///
		legend(on label(1 "Sample Size") label(2 "Smoking Rate")) ///
		name(fig`fcount', replace) saving(Figures\fig`fcount'.gph, replace)
	graph export Figures\fig`fcount'.eps, replace
restore
preserve
egen agetgroup = group(dmage tobacco)
collapse (mean) dbirwt (first) dmage tobacco, by(agetgroup)
local fcount = `fcount'+1
twoway scatter dbirwt dmage if tobacco==0 || scatter dbirwt dmage if tobacco==1|| , ///
		title("Average Birthweight vs. Age") ///
		xtitle("Age") ///
		ytitle("Birthweight") ///
		legend(on label(1 "Non-smokers") label(2 "Smokers")) ///
		name(fig`fcount', replace) saving(Figures\fig`fcount'.gph, replace)
	graph export Figures\fig`fcount'.eps, replace
restore

// Regression Table for:
//   dbirwt on tobacco and pscore 
//   dbirwt on tobacco, weighted by pscore ate
//   dbirwt on tobacco, weighted by pscore tot
//   dbirwt on tobacco, weighted by fractional tot
esttab r_bw_tp r_bw_tpw_ate_dbirwt r_bw_tpw_tot_dbirwt r_bw_tpw_tot200_dbirwt ///
	using TEXDocs/reg2.tex, replace ///
	order(_cons tobacco pscore tobacco_pscore) ///
	varlabels(_cons "Constant") ///
	cells(b(star fmt(%9.3f)) se(par fmt(%9.3f))) ///
	stats(r2 F N, fmt(%9.3f %9.3f %9.0g) ///
		layout("@" "@" "\multicolumn{1}{c}{@}") ///
		labels("\ensuremath{R^2}" "\ensuremath{F}" "\ensuremath{N}")) ///
	mtitles("OLS" "WLS (ATE)" "WLS (TOT)" "WLS (TOT$^{'}$)") ///
	label ///
	collabels(none) ///
	legend ///
	addnotes("Dependent variable is infant birthweight." ///
		"Robust standard errors in parentheses.") ///
	title("OLS and WLS Estimates of Birthweight Equations\label{reg:wls}") ///
	alignment(D{.}{.}{-1}) ///
	booktabs ///
	style(tex)

// Regression Table for:
//   lowbw on tobacco, weighted by pscore ate
//   lowbw on tobacco, weighted by pscore tot
esttab r_bw_tpw_ate_lowbw r_bw_tpw_tot_lowbw ///
	using TEXDocs/reg3.tex, replace ///
	order(_cons tobacco) ///
	varlabels(_cons "Constant") ///
	cells(b(star fmt(%9.3f)) se(par fmt(%9.3f))) ///
	stats(r2 F N, fmt(%9.3f %9.3f %9.0g) ///
		layout("@" "@" "\multicolumn{1}{c}{@}") ///
		labels("\ensuremath{R^2}" "\ensuremath{F}" "\ensuremath{N}")) ///
	mtitles("WLS (ATE)" "WLS (TOT)") ///
	label ///
	collabels(none) ///
	legend ///
	addnotes("Dependent variable is indicator for low birthweight." ///
		"Robust standard errors in parentheses.") ///
	title("WLS Estimates of Low Birthweight Equations\label{reg:wlslbw}") ///
	alignment(D{.}{.}{-1}) ///
	booktabs ///
	style(tex)

// Regression Table for:
//   death on tobacco 
//   death on tobacco and covariates
//   death on tobacco, weighted by pscore ate
//   death on tobacco, weighted by pscore tot
esttab r_d_t r_d_tc r_d_tpw_ate r_d_tpw_tot using TEXDocs/reg4.tex, replace ///
	order(_cons tobacco dmage dmage2 dmeduc dmeduc2 dmar dmblack ///
		dmhispan dmotherr alcohol nprevist disllb preterm pre4000 ///
		plural phyper diabete) ///
	varlabels(_cons "Constant") ///
	cells(b(star fmt(%9.6f)) se(par fmt(%9.6f))) ///
	stats(r2 F N, fmt(%9.6f %9.6f %9.0g) ///
		layout("@" "@" "\multicolumn{1}{c}{@}") ///
		labels("\ensuremath{R^2}" "\ensuremath{F}" "\ensuremath{N}")) ///
	mtitles("OLS" "OLS" "WLS (ATE)" "WLS (TOT)") ///
	label ///
	collabels(none) ///
	legend ///
	addnotes("Dependent variable is indicator for infant death." ///
		"Robust standard errors in parentheses.") ///
	title("OLS and WLS Estimates of Infant Death Equations\label{reg:death}") ///
	alignment(D{.}{.}{-1}) ///
	booktabs ///
	style(tex)

capture log close PS2
\end{verbatim}

\end{document}
